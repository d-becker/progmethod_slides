
\documentclass{beamer}
\usepackage{beamerthemeshadow}

\setbeamertemplate{headline}{%
\leavevmode%
  \hbox{%
    \begin{beamercolorbox}[wd=\paperwidth,ht=5ex,dp=1.125ex]{palette quaternary}%
    \insertsectionnavigationhorizontal{\paperwidth}{}{}
		\insertsubsectionnavigationhorizontal{\paperwidth}{}{\hskip0pt plus1filll}
    \end{beamercolorbox}%
  }
}

\begin{document}
\title{Programming methodology}  
\subtitle{Proving correctness of algorithms}
\author{Gergely Feldhoffer}
\date{ }



\frame{\titlepage} 

%\frame{\frametitle{Table of contents}\tableofcontents} 


\section{Course} 
\frame{
\frametitle{Overview of the course topic} 
\begin{itemize}
	\item Correctness of algorithms
	\item Proof of correctness
	\item Method to prove correctness
	\item Proofs for basic algorithms
	\item Some practical tools at the end of the semester concerning software correctness (gdb, valgrind)
\end{itemize}
}

\frame{
\frametitle{Course material} 
\begin{itemize}
	\item This material is originated from Dijsktra and Hoare using advanced mathematical toolkit as temporal logic
	\item We will discuss the same terms defined with basic set theory
	\item This model was created by {\'A}kos F{\'o}thi using Hungarian definitions
	\item
	$\rightarrow$ so be prepared, there are no English sources but this presentation
\end{itemize}

}


\frame{
\frametitle{What to do for your grade} 
\begin{itemize}
	\item At end of October or at the beginning of November you will write a test. The possible outcomes are
\begin{itemize}
	\item You are awarded with the best grade without taking the exam
	\item You can attend to the exam
	\item You have to write the test again at the last week
	
\end{itemize}
	\item At the exam you have to convince me about the clarity of your understanding of the theory of algorithm proving. 
\end{itemize}

}

\section{Introduction} 

\subsection{Sets}

\frame{
\frametitle{Relations}
\begin{itemize}
	\item You should be familiar with the terms of set, element, $\emptyset$, direct multiplication $A \times B$, elementary tuple $(x,y) \in A \times B$, relation $F \subseteq A \times B$, set size $\left|A\right|$
	\item $ \mathcal{D}_F = \{ \forall x \colon F(x) \neq \emptyset \} $ is the \textbf{domain} of the relation F
	\item $ \mathcal{R}_F = \{ \forall y \colon { \exists x \colon y \in F(x) } \} $ is the \textbf{range} of the relation F
	\item An $F$ relation is function if $\forall x \in \mathcal{D}_F \colon \left|F(X)\right| = 1$
\end{itemize} 
}

\subsection{State space}
\frame{\frametitle{State space}
\begin{block}{Def: State space}
Let $A_1, A_2, \ldots , A_n$ finite or countably infinite not empty sets. In this case $A=A_1 \times A_2 \times \ldots \times A_n$ is called \textbf{state space}, and $A_i$ are type value sets.
\end{block}

}

\subsection{Strings}
\frame{\frametitle{Strings}
\begin{itemize}
	\item Let $A$ be a set. We call all possible finite strings of $A$ as $A^*$. This operator is the Kleene star.
	\item If $\alpha \in A^*$ then the elements of the string can be expressed like $\alpha_1, \alpha_2, \ldots, \alpha_{|\alpha|}$
	\item The last element of the string can be also expressed like $\tau(\alpha)$
	\item $A^{\infty}$ is all possible infinite strings on set $A$.
	\item We will use $A^{**} = A^* \cup A^\infty$ for all finite and infinite string on set $A$
	\item The function $\tau$ is not defined on infinite strings.
	\item The reduction of a string is a process where every finite direct recurrences of the same elements are replaced with only one element, like $red(\langle1 2 3 3 3 4 4 5 \rangle)=\langle 1 2 3 4 5 \rangle$
\end{itemize}
}

\section{Program, Solution} 

\subsection{Problem}
\frame{\frametitle{Problem}
\begin{block}{Def.: Problem}
We will call $F \subseteq A \times A$ relation a \textbf{problem} on state space $A$
\end{block}

For a starting state $a \in A$ the possible valid solutions are the set $F(a)$. 

\begin{itemize}
	\item A relation is \textbf{deterministic} if $\forall a \in A \colon \left|F(a)\right| \leq 1$
	\item If a problem does not assign anything to an $a \in A$ that means that there are no demands for this starting state. A possible misunderstanding would be to think that there are no valid answers for the state, but we will use this state in quite an opposite way, so $F(a) = \emptyset$ and $F(a)=A$ are equivalent in some way.
\end{itemize}

}

\subsection{Program}

\frame{\frametitle{Program}
\begin{block}{Def.: Program}
$S \subseteq A \times A^{**}$ is program if \\
\begin{enumerate}
	\item $\mathcal{D}_S = A$
	\item $\forall a \in A : \forall \alpha \in S(a) \colon \alpha_1=a$
	\item $\forall \alpha \in \mathcal{R}_S \colon \alpha = red(\alpha) $
\end{enumerate} 
\end{block}

\begin{itemize}
	\item The strings of states are the temporal stages of the machine as the program runs. Different starting states can mean different input values.
	\item The program has to start from every state
	\item The program must start in the starting state
	\item In each step something has to happen

\end{itemize}

}

\frame{\frametitle{Program and variables}

\begin{block}{Def: Variable}
Let $A=A_1 \times A_2 \times \ldots \times A_n$ be a state space. In this case for each $i \in [1 \ldots n] : f_i \subseteq A \times A_i = \{ x \in A \colon f_i(x)=proj_{A_i}(x)  \} $
\end{block}

So the variable is not a single one dimensional subspace of the state space but the projection function to it. The subspace is the set containing the possible elements eg. integer numbers. The variable is more than a set, it has history as the program runs. For each $S(a)$ a projection of this string is how the given variable changes its value.

}


\subsection{Program function}
\frame{\frametitle{Program function}
\begin{block}{Def.: Program function}
If $S \subseteq A \times A^{**}$ is program then we will define $p(S) \subseteq A \times A$ \textbf{program function} of $S$ as
\begin{enumerate}
	\item $\mathcal{D}_{p(S)} = \{ a \in A \colon S(a) \subseteq A^* \}$
	\item $\forall a \in \mathcal{D}_{p(S)} \colon p(S)(a) = \{ b \in A \colon { \exists \alpha \in S(a) \colon b = \tau(\alpha) } \} $
\end{enumerate}
\end{block}
\begin{itemize}
	\item So the program function is defined only where the program certainly terminates,
	\item and assigns to starting state $a$ the set of the possible outcomes of that state $a$ at the end of the program run.
\end{itemize}

}

\subsection{Solution}
\frame{\frametitle{Solution}
\begin{block}{Def.: Solution}
An $S \subseteq A \times A^{**}$ program \textbf{solves} the problem $F$ if
\begin{enumerate}
	\item $\mathcal{D}_F \subseteq \mathcal{D}_{p(S)}$
	\item $\forall a \in \mathcal{D}_F \colon p(S)(a) \subseteq F(a) $
\end{enumerate}
\end{block}

\begin{itemize}
	\item So the program should certainly terminate where there are any demands for a starting state,
	\item and for each starting state the set of possible outcomes of the program run must be ones of the valid outcomes.
	\item if the program is deterministic then this can be expressed as $p(S)(a) \in F(a)$, and if both the program and the problem is deterministic, then $p(S)(a) = F(a)$
\end{itemize}

}




\section{Specification}
\subsection{Precondition, postcondition}
\frame{\frametitle{Truth set}
If $R \subseteq A \times \mathbb{L}$ then
\begin{itemize}
	\item 
\textbf{Truth set} of R : \\
$\lceil R \rceil = \{ a \in A \colon R(A) = \{true\} \}$
	\item 

\textbf{Weak truth set} of R : \\
$\lfloor  R \rfloor = \{ a \in A \colon true \in R(A) \}$
	\item 

Note that if $R$ is function then $\lceil R \rceil = \lfloor R \rfloor$	
	\item 

$Q \rightarrow R \Rightarrow \lceil Q \rceil \subseteq \lceil R \rceil$
\end{itemize}

}


\frame{\frametitle{Precondition, postcondition}

\begin{itemize}
	\item 

We will use the term \textbf{precondition} as a condition which is true for all starting state of the state space where we have demand for the output, and the 

  \item

\textbf{postcondition} which is true for all valid input-output pairs and false for everything else.

	\item 

If $S \subseteq A \times A^{**}$ is a program any $R \subseteq A \times \mathbb{L}$ condition on state space $A$ can be used as precondition or postcondition

	\item
Our goal is to express the problem using only precondition and postcondition.

\end{itemize}

}

\subsection{Weakest precondition}
\frame{\frametitle{Weakest precondition}
\begin{block}{Def: Weakest precondition}
$S \subseteq A \times A^{**}$ is a program, $R \subseteq A \times \mathbb{L}$ is a condition. $R$ is the \textbf{weakest precondition} of $S$ if

$\lceil wp(S,R) \rceil = \{ a \in \mathcal{D}_{p(S)} \colon
p(S)(a) \subseteq \lceil R \rceil \}$

\end{block}

So which is the section of the state space where if we run $S$ program, we will certainly arrive in states where $R$ is true.

}




\frame{\frametitle{Properties of weakest precondition}
Let $S \subseteq A \times A^{**}$ be a program, and $Q,R
\subseteq A \times \mathbb{L}$,

then

\begin{itemize}
\item
Principle 'Exclusion of miracles': $wp(S,False) = False$ where $False \subseteq A \times \mathbb{L} = \forall a \in A \colon False(a) = false $

\item

Monotonicity: If $Q \Rightarrow R$ then $wp(S,Q) \Rightarrow wp(S,R)$



	\item $wp(S,Q) \wedge wp(S,R) = wp(S, Q \wedge R) $
	\item $wp(S,Q) \vee wp(S,R) = wp(S, Q \vee R) $
\end{itemize}
}


\subsection{Parameter space}
\frame{
\frametitle{Parameter space}

\begin{block}{Def: Parameter space}
Let $F \subseteq A \times A$ problem. We name set $B$ \textbf{parameter space} of $F$ if $\exists F_1, F_2 \colon F_1 \subseteq A \times B \wedge F_2 \subseteq B \times A \wedge F = F_2 \circ F_1$ 
\end{block}

where $R_1 \circ R_2$ is composition of relations meaning if 
$a \in (R_1 \circ R_2)(b)$ then $a \in R_1(R_2(b))$

note that the order of arguments of expression $R_1 \circ R_2$ and the order of evaluation is opposite.

The parameter space can be subspace of the state space, restricting to the input variables. This will be our choice from now on, however, this definition allows to use even the whole state space.
}


\subsection{Specification}
\frame{\frametitle{Specification}
\begin{block}{Theorem: Theorem of Specification}
Let $F \subseteq A \times A$ problem, $B$ is a parameter space of $F$, $F_1 \subseteq A \times B \wedge F_2 \subseteq B \times A \wedge F = F_2 \circ F_1$, and $b \in B$

$\lceil Q_b \rceil = \{ a \in A \colon (a,b) \in F_1 \} = inv(F_1)(b)$

$\lceil R_b \rceil = \{ a \in A \colon (b,a) \in F_2 \} = F_2(b)$

In this case if $\forall b \in B \colon Q_b \Rightarrow wp(S,R_b)$ then program $S$ solves problem $F$

\end{block}
\vfill

Which means if we can express our demands with precondition and postcondition, then we defined a problem, and we also give a possible method to check whether a program is a solution or not.

}

\frame{\frametitle{Proof of theorem of specification}
Let $F \subseteq A \times A$ problem, $B$ is a parameter space of $F$, $F_1 \subseteq A \times B \wedge F_2 \subseteq B \times A \wedge F = F_2 \circ F_1$, and $b \in B$

$\lceil Q_b \rceil = \{ a \in A \colon (a,b) \in F_1 \} = inv(F_1)(b)$

$\lceil R_b \rceil = \{ a \in A \colon (b,a) \in F_2 \} = F_2(b)$

In this case if $\forall b \in B \colon Q_b \Rightarrow wp(S,R_b)$ then program $S$ solves problem $F$

Proof:
\begin{itemize}
	\item 

$\mathcal{D}_F \subseteq \mathcal{D}_{p(S)}$ since every $a \in \mathcal{D}_F \colon \exists b \in B \colon a \in \lceil Q_b \rceil$ because of $F_1$. Since $Q_b \Rightarrow wp(S,R_b)$ so $a \in \lceil Q_b \rceil \subseteq \lceil wp(S,R_b) \rceil \subseteq \mathcal{D}_{p(S)}$

\item
$\forall a \in \mathcal{D}_F \colon p(S)(a) \subseteq F(a)$ since if $a \in \mathcal{D}_F$ and $b \in B \wedge a \in \lceil Q_b \rceil$ then
$p(S)(a) \subseteq \lceil R_b \rceil = F_2(b) \subseteq F_2(F_1(a)) = F(a)$

\end{itemize}

}

\subsection{Example}
\frame{\frametitle{Example of specification}
Example: addition of two numbers

$A = \mathbb{Z}_x \times \mathbb{Z}_y \times \mathbb{Z}_z $

$B = \mathbb{Z}_{x'} \times \mathbb{Z}_{y'} $

$Q_{x'(b), y'(b)}(a) = (x(a) = x'(b) \wedge y(a) = y'(b))$

$R_{x'(b), y'(b)}(a) = (x(a) = x'(b) \wedge y(a) = y'(b) \wedge z(a)=x'(b)+y'(b))$

\vfill
We will use $x'$ and $y'$ are represent the starting value of the variable as it is in the parameter space.

\vfill
This formalism can be simplified since each raw variable is defined on $A$ and each parameter is defined on $B$ ..

}

\frame{\frametitle{Example of specification}
Example: addition of two numbers

$A = \mathbb{Z}_x \times \mathbb{Z}_y \times \mathbb{Z}_z $

$B = \mathbb{Z}_{x'} \times \mathbb{Z}_{y'} $

$Q_{x', y'}(a) = (x = x' \wedge y = y')$

$R_{x', y'}(a) = (x = x' \wedge y = y' \wedge z=x'+y')$

\vfill

..and since $Q$ and $R$ has to be defined on the whole parameter space..
}

\frame{\frametitle{Example of specification}
Example: addition of two numbers

$A = \mathbb{Z}_x \times \mathbb{Z}_y \times \mathbb{Z}_z $

$B = \mathbb{Z}_{x'} \times \mathbb{Z}_{y'} $

$Q : x = x' \wedge y = y'$

$R : x = x' \wedge y = y' \wedge z=x'+y'$

\vfill

..we arrived to the formalism we will use for specifications.
}



\section{Elementary programs}
\subsection{Elementary program}
\frame{\frametitle{Elementary program}

\begin{block}{Elementary program}

$S \subseteq A \times A^{**}$ program is \textbf{elementary} if $\forall a \in A \colon \forall \alpha \in S(a) \colon ( \exists b \in A \colon \alpha = red( \langle  a b \rangle ) ) \vee \alpha = \langle a a \ldots \rangle$

\end{block}
So a program is elementary if it does exactly one step or hangs.

We will use some special elementary programs such as SKIP, ABORT, and assignment.


}

\subsection{Skip Abort}
\frame{\frametitle{Skip Abort}
\begin{block}{SKIP}
$SKIP = \{ (a,\alpha) \in A \times A^{**} \colon \alpha = \langle a \rangle  \} $
\end{block}
\begin{block}{ABORT}
$ABORT = \{ (a,\alpha) \in A \times A^{**} \colon \alpha = \langle a a a \ldots \rangle  \} $

\end{block}
}


\subsection{Assignment}
\frame{\frametitle{Assignment}
Every other elementary programs are general assignments. To handle assignments, we will use a state transition function $E$ to describe it: $E \subseteq A \times A$, so if we are in state $a$ we step to (one of the) state(s) $E(a)$.
\vfill
So if we have an $E$ as $\mathcal{D}_E = A$, then \\
$S_E = \{ (a,\alpha) \in A \times A^{**} \colon \alpha = red(\langle a b \rangle) \colon b \in E(a) \}$
}

\frame{
\frametitle{Assignment} 

In general $\mathcal{D}_E = A$ is not true, in these cases where the state transition function is not defined, the assignment should abort.
\begin{block}{Assignment}
$ S_E = \{ (a,\alpha) \in A \times A^{**} \colon a \in \mathcal{D}_E \wedge \alpha = red(\langle a b \rangle) \wedge b \in E(a) \} \cup $\\$ \cup \{ (a,\alpha) \in A \times A^{**} \colon a \notin \mathcal{D}_E \wedge \alpha = \langle a a a \ldots \rangle \} $
\end{block}

Note that every state transition can be an assignment. Each machine support a limited possibilities of state transitions, that is why we can not solve NP complete tasks in one step.

Computer science calls esoteric imaginary state transitions as oracle. Interestingly it is proven that on a machine which supports every usual operation and includes one NP complete task, on that machine P=NP. The hard part is to prove that NP on this machine is not stronger.
}

\subsection{Program functions}
\frame{\frametitle{Program functions of elementary programs}
$wp(SKIP,R) = R$
\vfill
$wp(ABORT,R) = False$
\vfill
$wp(S_E, R) = \{ a \in \mathcal{D}_E \colon E(a) \subseteq \lceil R \rceil \}$
\vfill
$\mathcal{D}_{p(SKIP)} = A$ \\
$\forall a \in A \colon p(SKIP)(a) = a$
\vfill
$\mathcal{D}_{p(ABORT)} = \emptyset$
\vfill
$p(S_E) = E$
}


\frame{\frametitle{Notations}
Example:
$A = \mathbb{Z}_x \times \mathbb{Z}_y $\\
$ E= \{ (a,b) \in A \times A \colon y(b)=2*x(a) \wedge x(b)=x(a) \} $
\\
We will be noted as $y := 2x$, this is a simple assignment
\vfill
$ E= \{ (a,b) \in A \times A \colon y(b)=2*x(a) \wedge x(b)=x(a)+1 \} $
\\
as $x, y :=x+1, 2x$, this is a parallel assignment
\vfill
$ E= \{ (a,b) \in A \times A \colon y(b) \in X \} $
\\
as $y :\in X$, this is a value choice

}


\section{Structures}
\subsection{Sequence}
\frame{\frametitle{Program structures}
We will use three kind of program structure: sequence, branch, and loop.
\vfill
The sequence means running two program after each other. The branch means state dependent conditional program execution. The loop is state dependent repeated execution of a program.
\vfill
Note that every program can be used as part of program structure. Every usage of a program structure results a program.
}

\frame{\frametitle{Sequence}
\begin{block}{Def: Sequence}
Let $S_1, S_2 \subseteq A \times A^{**}$ be programs. We will call $S \subseteq A \times A^{**}$ as \textbf{Sequence of $S_1$ and $S_2$} if $\forall a \in A \colon$ \\
$S(a) = \{ \alpha \in A^\infty \colon \alpha \in S_1(a) \} \cup$ \\
$\cup \{ red(concat(\alpha, \beta)) \in A^{**} \colon \alpha \in S_1(a) \cap A^* \wedge \beta \in S_2(\tau(\alpha)) \}$
\end{block}
}

\subsection{Branch}
\frame{\frametitle{Branch}
\begin{block}{Def: Branch}
Let $\pi_1, \ldots, \pi_n: A \rightarrow \mathbb{L}$ be conditions, $S_1, \ldots, S_n$ programs on $A$.
We will call $IF \subseteq A \times A^{**}$ program as \textbf{Branch} of $(\pi_1:S_1, \ldots, \pi_n:S_n)$ if $\forall a \in A$:
\[
IF(a) = \bigcup_{i=1}^{n} w_i(a) \cup w_0(a)
\]

\[
\textrm{where }  \forall i \in [ 1 .. n ] \colon 
w_i(a) = \left\{
\begin{array}{ll}
S_i(a)    &  \textrm{if } \pi_i (a),\\
\emptyset &  \textrm{else} 
\end{array}
\right.
\]

\[
w_0(a) = \left\{
\begin{array}{ll}
\langle a a a \ldots \rangle    & \textrm{if } \forall i \in [1 .. n ] \colon \neg \pi_i (a),
\\
\emptyset &  \textrm{else} 
\end{array}
\right.
%
\]

\end{block}

}

\frame{\frametitle{Branch}
\begin{itemize}
	\item Note that a branch can be non-deterministic: if more than one condition are true in a state each of the routes can happen.
	\item In states where no branch condition is true, the branch must abort. This is not the usual approach, as C switch without default performs SKIP instead of ABORT.
\end{itemize}
}

\subsection{Loop}
\frame{\frametitle{Loop}
\begin{block}{Def: Loop}
Let $\pi$ be a condition and $S_0$ a program on $A$. We will call $DO \subseteq A \times A^{**}$ a \textbf{Loop} if
\begin{itemize}
	\item $\forall a \notin \lceil \pi \rceil \colon DO(a) = \{ \langle a \rangle \}$
	\item $\forall a \in \lceil \pi \rceil \colon DO(a) =$
\end{itemize}
\[
\begin{array}{lll}
& \{\alpha \in A^{**} \colon \exists \alpha^1, \ldots, \alpha^n \in A^{**} \colon & (1) \\

& \alpha = red(concat(\alpha^1, \ldots, \alpha^n)) \wedge \alpha^1 \in S_0(a) \wedge & (2) \\

& \wedge ( \forall i \in [ 1 .. n-1 ] \colon \alpha^i \in A^* \wedge & (3) \\

& \wedge \alpha^{i+1} \in S_0(\tau(\alpha^i)) \wedge \pi(\tau(\alpha^i))) \wedge & (4) \\

& \wedge (( \alpha^n \in A^\infty) \vee (\alpha^n \in A^* \wedge \neg \pi(\tau(\alpha^n))))\} & (5) \\

\cup & \{ a \in A^\infty \colon \forall i \in \mathbb{N} \colon & (6) \\

& \exists \alpha^i \in A^* \colon \alpha = red(concat(\alpha^1, \alpha^2, \ldots)) \wedge & (7) \\

& \alpha^1 \in S_0(a) \wedge (\forall i \in \mathbb{N} \colon \alpha^i \in A^* \wedge & (8) \\

& \alpha^{i+1} \in S_0(\tau(\alpha^i)) \wedge \pi(\tau(\alpha^i)))\} & (9) \\

\end{array}
\]
\end{block}

}

\frame{\frametitle{Loop}
\begin{itemize}
	\item 
So the loop is the repetition of the loop body where the resulting program string is a concatenation of the iterations (2) where the iterations are finite in the first $n-1$ cases (3), all of the iterations starts where the previous one ended (4) and the loop condition is true(4), and the end of this line may be infinite or such kind of finite where the end of the last iteration goes outside of the loop conditions truth set(5),
\item or, the string of the loop can be endless without aborting if there are infinite number of finite iterations where the loop condition is always true for the end of the strings of iterations (6-9).
\end{itemize}

}
\subsection{Program functions of program structures}
\frame{\frametitle{Program function of sequence}
Composition of relations $R_1 \subseteq A \times B, R_2 \subseteq B \times C$ : $R_2 \circ R_1 = \{ (a,c) \in A \times C \colon \exists b \in B \colon (a,b) \in R_1 \wedge (b,c) \in R_2 \}$
\vfill
Strict composition of relations $R_1 \subseteq A \times B, R_2 \subseteq B \times C$ : $R_2 \odot R_1 = \{ (a,c) \in A \times C \colon \exists b \in B \colon (a,b) \in R_1 \wedge (b,c) \in R_2 \wedge R_1(a) \subseteq \mathcal{D}_{R_2}\}$
\vfill

\begin{block}{Theorem: Program function of sequence}
If $S=(S_1;S_2)$ sequence of $S_1, S_2$ programs\\
$p(S) = p(S_1) \odot p(S_2)$
\end{block}

}

\frame{\frametitle{Program function of branch}
\begin{block}{Theorem: Program function of branch}
If $IF(\pi_1:S_1, \ldots, \pi_n:S_n) \subseteq A \times A^{**}$ is a branch, then \\
$\mathcal{D}_{p(IF)}=\{ a \in A \colon \bigvee_{i=1}^n \pi_i(a) \wedge \forall i \in [1 \ldots n ] \colon \pi_i(a) \Rightarrow a \in \mathcal{D}_{p(S_i)} \}$

\[\forall a \in \mathcal{D}_{p(IF)} \colon p(IF)(a) = \bigcup_{i=1}^n pw_i(a)\]


\[
\textrm{where }  \forall i \in [ 1 .. n ] \colon 
pw_i(a) = \left\{
\begin{array}{ll}
p(S_i)(a)    &  \textrm{if } \pi_i (a),\\
\emptyset &  \textrm{else} 
\end{array}
\right.
\]


\end{block}
So the program function of IF is defined in states where at least one condition is true, and no $S_i$ hangs there, so it will certeanly terminate, and every possible pathway is included where the condition is true.
}

\frame{\frametitle{Program function of loop}
Any homogeneous $R$ relation can be composed with itself. $R^n = R \circ R^{n-1}$ where $R^0$ is the identity.
\begin{block}{Def: Exclosure of relation}
If $R \subseteq A \times A$ is a relation, the exclosure of $R$ is 
$\bar{R} \subseteq A \times A$ where \\
$\mathcal{D}_{\bar{R}} = \{ a \in A \colon \nexists \alpha \in A^\infty \colon \alpha_1 \in R(A) \wedge \forall i \in \mathbb{N} \colon \alpha_{i+1} \in R(\alpha_i) \} $

$\forall a \in \mathcal{D}_{\bar{R}} \colon \bar{R}(a) = \{ b \in A | \exists k \in \mathbb{N}_0 \colon b \in R^k(a) \wedge b \notin \mathcal{D}_{R}  \} $
\end{block}

So the resulting relation contains every $(a,b)$ pair where there is a path to reach $b$ from $a$, and there is no step further from $b$ available.
}

\frame{\frametitle{Program function of loop}
If $R \subseteq A \times A $ and $B \subset A$ then $R|_B = \{ (a,b) \in B \times A \colon (a,b) \in R\}$
This is the restriction of the relation to a subset.

\begin{block}{Def: Restriction to condition}
If $R \subseteq A \times A $ and $\pi \subseteq A \times \mathbb{L}$ then $R|_\pi = (R \cap (\lceil \pi \rceil \times A)) \cup \{ (a,a) \in A \times A \colon a \in \lceil \pi \rceil \setminus \mathcal{D}_R \}$

\end{block}
So the restriction to a condition differs from the restriction to the truth set of the condition as it is broadened with identities, making $\mathcal{D}_{R|_\pi}=\lceil \pi \rceil$
}

\frame{\frametitle{Program function of loop}
\begin{block}{Theorem: Program function of loop}
If $DO(S_0,\pi)$ is a loop then\\
$p(DO) = \overline{p(S_0)|_\pi}$
\end{block}
}




\section{Deduction}
\subsection{Deduction rules}
\frame{\frametitle{Deduction rules}
We will give a set of deduction rules for program structures. As every program can be expressed as a hierarchical structure where the nodes of the hierarchy are program structures and the leaves are elementary programs, rules for each structure is enough to deduce every program.

}

\subsection{Sequence}
\frame{\frametitle{Deduction rules for sequence}
\begin{block}{Deduction rules for sequence}
Let $S=(S_1;S_2)$ be a sequence, and $Q,R,Q' \subseteq A \times \mathbb{L}$ statements on A. If
\begin{enumerate}
	\item $Q \Rightarrow wp(S_1, Q')$ and
	\item $Q' \Rightarrow wp(S_2, R)$, 
\end{enumerate}
then $Q \Rightarrow wp(S,R)$.
\end{block}

}

\subsection{Branch}
\frame{\frametitle{Deduction rules for branch}
\begin{block}{Deduction rules for branch}
Let $IF=(\pi_1:S_1 \ldots \pi_n:S_n)$ be a branch and $Q,R \subseteq A \times \mathbb{L}$ statements on A. If
\begin{enumerate}
	\item $Q \Rightarrow \bigvee_{i=1}^n \pi_i$, and
	\item $\forall i \in [1 \ldots n] \colon Q \wedge \pi_i \Rightarrow wp(S_i, R)$,
\end{enumerate}
then $Q \Rightarrow wp(IF, R)$.
\end{block}

}

\subsection{Loop}
\frame{\frametitle{Deduction rules for loop}
\begin{block}{Deduction rules for loop}
Let $DO(\pi, S_0)$ be a loop, $P,Q,R \subseteq A \times \mathbb{L}$ statements and $t: A \rightarrow \mathbb{Z}$ function. If
\begin{enumerate}
	\item $Q \Rightarrow P$
	\item $P \wedge \neg \pi \Rightarrow R$
	\item $P \wedge \pi \Rightarrow wp(S_0, P)$
	\item $P \wedge \pi \Rightarrow t>0$
	\item $P \wedge \pi \wedge t=t_0 \Rightarrow wp(S_0, t<t_0)$
\end{enumerate}
then $Q \Rightarrow wp(SO,R)$
\end{block}
}

\frame{\frametitle{Deduction rules for loop}
$P$ is the loop invariant, and $t$ is the terminating function. 

$P$ must be true all along the loop and even after the last iteration. $P$ definitely not equals $\pi$, as the second role shows. $P$ is the connection between the start and the end of the loop iteration chain.

The hard part of proving the correctness of a program is to guess $P$.

$t$ should be used to confirm that the loop is not infinite. The value of $t(a)$ must be decreased along the loop body borders in each iteration, and must be positive. As an integer function infinite regression cannot happen.
}

\frame{\frametitle{behold, adventurer}
no slides beyond
}


\section{Recursion}
\frame{\frametitle{todo}
todo
}

\section{Tools}
\frame{\frametitle{todo}
todo
}


%\frame{\frametitle{lists with pause}
%\begin{itemize}
%\item Introduction to  \LaTeX \pause 
%\item Course 2 \pause 
%\item Termpapers and presentations with \LaTeX \pause 
%\item Beamer class
%\end{itemize} 
%}
%
%\subsection{Lists II}
%\frame{\frametitle{numbered lists}
%\begin{enumerate}
%\item Introduction to  \LaTeX  
%\item Course 2 
%\item Termpapers and presentations with \LaTeX 
%\item Beamer class
%\end{enumerate}
%}
%\frame{\frametitle{numbered lists with pause}
%\begin{enumerate}
%\item Introduction to  \LaTeX \pause 
%\item Course 2 \pause 
%\item Termpapers and presentations with \LaTeX \pause 
%\item Beamer class
%\end{enumerate}
%}
%
%\section{Section no.3} 
%\subsection{Tables}
%\frame{\frametitle{Tables}
%\begin{tabular}{|c|c|c|}
%\hline
%\textbf{Date} & \textbf{Instructor} & \textbf{Title} \\
%\hline
%WS 04/05 & Sascha Frank & First steps with  \LaTeX  \\
%\hline
%SS 05 & Sascha Frank & \LaTeX \ Course serial \\
%\hline
%\end{tabular}}
%
%
%\frame{\frametitle{Tables with pause}
%\begin{tabular}{c c c}
%A & B & C \\ 
%\pause 
%1 & 2 & 3 \\  
%\pause 
%A & B & C \\ 
%\end{tabular} }
%
%
%\section{Section no. 4}
%\subsection{blocs}
%\frame{\frametitle{blocs}
%
%\begin{block}{title of the bloc}
%bloc text
%\end{block}
%
%\begin{exampleblock}{title of the bloc}
%bloc text
%\end{exampleblock}
%
%
%\begin{alertblock}{title of the bloc}
%bloc text
%\end{alertblock}
%}

\end{document}
